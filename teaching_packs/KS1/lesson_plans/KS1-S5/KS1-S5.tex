\documentclass{../../../lessonplan}
\renewcommand{\cflroot}{../../..}

\begin{document}

\lessonplantitle
    {KS1-S5}
    {Key Stage 1 Session 5}
    {Create a more complex algorithm to deliver one or more packages along the way}

\preamble
    {
    \item Write an algorithm to include intermediate deliveries 
    }
    {
    \item Levels 15 to 16 in Rapid Router
    \item Resource sheet KS1-S5-1 (laminated if possible)
    \item Grocery card and van card
    }
    {
    \item Directly, on the way
    \item Destination, delivery, deliver
    }

\begin{lessonplan}

Talk to the children about delivery vans which have to stop off at different points and delivery packages.
Discuss the types of services (grocery, clothes, books etc.)

Demonstrate this using resource sheet KS1-S5-1 \textit{[fig S5.1]}, a grocery card \textit{[fig S5.2]}, and a van card \textit{[fig S5.3]}. 

\fig{fig S5.1}{figS5.1.jpg}{1}

\fig{fig S5.2}{figS5.2.jpg}{1}

\fig{fig S5.3}{figS5.3.jpg}{1}

Introduce level 15 of the app \textit{[fig S5.4]}, and the new \keyword{deliver} command.
The van has to travel from the warehouse to deliver some groceries at one house before continuing to the final house.

\fig{fig S5.4}{figS5.4.jpg}{1}

\keyquestion{Which path could the van take?} Discuss what choices the van driver could have, and which would be the best (the shortest, most efficient path).

\subsection*{Independent activity}

Children work through levels 15 and 16.

\subsection*{Share and review}

Discuss the pressures on delivery vans in real life.
\keyquestion{If you have five places to deliver packages to in a day, will it matter which order you do it?}

Explain that they will code that type of journey in the next session.

\end{lessonplan}

\end{document}
