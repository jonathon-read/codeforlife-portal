\documentclass{../../../lessonplan}
\renewcommand{\cflroot}{../../..}

\begin{document}

\lessonplantitle
    {KS1-S4}
    {Key Stage 1 Session 4}
    {Creating simple algorithms to reach a single destination along the shortest route}

\preamble
    {
    \item Identify different algorithms to reach the same destination
    \item Select the most efficient algorithm and create the code for this
    \item Begin to debug a sequence of instructions
    }
    {
    \item Levels 13 to 14 in Rapid Router
    \item Laptops with app or portal address bookmarked
    \item Projector or Interactive Whiteboard (IWB)
    \item Resource sheet KS1-S4-1 (laminated if you wish to re-use), KS1-S4-2 and KS1-S4-3
    }
    {
    \item Shortest, longest
    \item Route
    }

\begin{lessonplan}

Display level 13 of Rapid Router on the IWB \textit{[fig S4.1]}.

\fig{fig S4.1}{figS4.1.jpg}{1}

\keyquestion{Is there more than one way to get to the house?} 
In pairs, give the children a printed copy of pupil resource sheet KS1-S4-1 to discuss and work out the possibilities.

Ask a child to trace their route on the IWB.

\textbf{Discuss:} In real life, van drivers need to be efficient and get their deliveries to the houses as quickly as possible.
This usually means taking the shortest route.

\keyquestion{Which do you think is the shortest route?}

\keyquestion{How can we work out the distance the van travels?}
(You could count this in terms of the grid squares travelled through).

Work out the total distance for the route.
Children can record this on a whiteboard.

\keyquestion{Look at a different route. Is it longer or shorter? How can we check?}

Together, builder the code for the route the class thinks is the shortest.

\section*{Individual or paired activity}

Children use levels 13 and 14 of Rapid Router for this activity.
Discuss which the shortest route is each time.

Note---if some children need a further challenge, go to the create mode and set up some new routes for them to navigate.
Save the routes and share with the class \textit{[fig S4.2]}.

\fig{fig S4.2}{figS4.2.jpg}{1}

Alternatively use resource sheet KS1-S4-2 \textit{[fig S4.3]}, which has similar but different routes.

\fig{fig S4.3}{figS4.3.jpg}{1}

\section*{Share and review}

Explain the code for an inefficient route at level 14 \textit{[fig S4.4]} using the IWB.

\fig{fig S4.4}{figS4.4.jpg}{1}

Ask a pair of children to come up and improve the program, so that the van travels a shorter or the shortest route.

\keyquestion{Using the world `code', can you explain what you have done?}

Talk about how much shorter the new route is.

\section*{Assessment}

Look at resource sheet KS1-S4-3 \textit{[fig S4.5]}.

\fig{fig S4.5}{figS4.5.jpg}{1}

\keyquestion{How many different routes can you spot? Which is the shortest?}

Discuss what they have learnt this sessions.

\section*{Extension follow-up}

Use resource sheet KS1-S4-4 to investigate how many different routes you can find using a distance of 10 grid squares to go from A to B \textit{[fig S4.6].}

\fig{fig S4.6}{figS4.6.jpg}{1}

\section*{Cross-curricular links}

\subsection*{Mathematics}

\begin{itemize}
\item Number and computation---consolidating number bonds to 20
\item Mathematical thinking---following the unplugged extension activity, some children will understand the concept that the total distance travelled is the sum of all the horizontal distances plus the sum of the vertical distances
\item Two different routes may cover the same distance but would they take the same time in real life?
\item If you have to make more turns in one route it may take more time
\end{itemize}

\subsection*{Literacy}

\begin{itemize}
\item Talk about the stories you have read where vehicles make journeys -- Postman Pat, the Jolly Postman etc.
\end{itemize}

\subsection*{Geography}

\begin{itemize}
\item Talk about how we use plans and maps to find our way from one place to another and look at a plan of the school as an example.
\end{itemize}

\end{lessonplan}

\end{document}
