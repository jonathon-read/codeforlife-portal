\documentclass{../../../lessonplan}
\renewcommand{\cflroot}{../../..}

\begin{document}

\lessonplantitle
    {KS1-S9}
    {Key Stage 1 Session 9}
    {Children create their own routes}

\preamble
    {
    \item Design a programming challenge for a friend
    \item Use logical reasoning to check that the challenge is achievable
    }
    {
    \item Levels 26 to 28 in Rapid Router  
    \item Create mode in Rapid Router
    \item Resource sheets KS1-S9-1 for children to plan their routes and record their challenge instructions
   \item Interactive Whiteboard (IWB)
    }
    {
    \item Design, program
    \item Create, debug
    \item Predict, repeat
    \item Loop, challenge
    }

\begin{lessonplan}

Explain that now the children now how to program efficient routes, they are going to create their own roads and scenery.
They can also choose a character to travel along the road.

Show levels 26 to 28 as examples of different scenes that can be created \textit{[fig S9.1]}.

\fig{fig S9.1}{figS9.1.jpg}{1}

Recap how to log on to their own accounts so they can save their scenarios.

Demonstrate how you can use the scene tiles to design your own environment, and place houses to deliver the packages.
Make sure the children understand how to set the start and end blocks \textit{[fig S9.2]}.

\fig{fig S9.2}{figS9.2.jpg}{1}

\section*{Individual activity}

Explain that you would like the children to design their own environment and select objects to deliver.
They should think of a story to go with their route, for example where the van is going and who it is delivery to.

Some children will think of simple routes, others more complex routes.

Demonstrate how to write instructions for a friend to explain the challenge, using resource sheet KS1-S9-1 \textit{[fig S9.3]}.
It could be simply `drive from the warehouse to the house'. 
It might be `drive to the house, but stop at the cafe on the way', and so on.

\fig{fig S9.3}{figS9.3.jpg}{1}

Children must test out the challenge, save the project and write their project name on their sheet.

\section*{Share and review}

Choose a few children to describe their challenges, ready for their partners to try in the next session \textit{[fig S9.4]}.

\fig{fig S9.4}{figS9.4.jpg}{1}

\keyquestion{What is the story of your route?}

\keyquestion{What is the van delivering?}

\keyquestion{Can you explain why you chose your background?}

\section*{Extension}

You could link this activity to other curricular topics, using the range of characters and objects in the app, e.g.\ re-telling or re-working a story that has a journey in it.

For example, after reading Handa's Surprise, a character goes around the path collecting items of fruit, or the wolf might go along the road past the woodcutter's cabin, to Granny's house.

\section*{Cross-curricular links}

\subsection*{Literacy, Geography, Mathematics}

There are many ways to link this activity with other subjects.
For example, geography links with routes around your local area, which you could show in satellite form on Google maps.

\end{lessonplan}

\end{document}
