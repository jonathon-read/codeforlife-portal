\documentclass{../../../lessonplan}
\renewcommand{\cflroot}{../../..}

\begin{document}

\lessonplantitle
    {KS1-S5}
    {Key Stage 1 Session 6}
    {Create a more complex algorithm to deliver one or more packages along the way}

\preamble
    {
    \item Write an algorithm to include intermediate deliveries 
    }
    {
    \item Levels 17 to 18 in Rapid Router
    \item Levels 17 and 18 of Rapid Router from the Levels Guide
    \item Resource sheets KS1-S6-1 and KS1-S6-2
    }
    {
    \item Directly, on the way
    \item Deliver, destination
    }

\begin{lessonplan}

Explain that you are going to look at more complicated routes for delivery vans.

\keyquestion{What if you have more than one package to deliver before arriving at your final destination?}

Give two children a basket or a bag each.
Ask them to stand in different parts of the classroom.

Demonstrate the activity by pretending to be the van driver holding two packages.
Take a long and inefficient route to deliver the packages, and then reach your final destination.
Ask the pair to record the movements using the \keyword{forward}, \keyword{turn}, and \keyword{deliver} instructions.

Ask the children to comment on how good your route was.

Now get them to `do better'.
Position the children in different places.

Ask one child to be the van driver, holding the two packages.

Ask the others to give instructions to the van driver to deliver each package in turn whilst another child or pair records the journey.

\textbf{Discuss:} \keyquestion{How did you decide which package to \keyword{deliver} first?}

\section*{Paired activity}

Using resource sheets KS1-S6-1 and KS1-S6-2, ask the children to try the challenges involving one or more stops on the way to the house (levels 17 \textit{[fig S6.1]} to 18 \textit{[fig S6.2]}).

\fig{fig S6.1}{figS6.1.jpg}{1}

\fig{fig S6.2}{figS6.2.jpg}{1}

\section*{Share and review} 

Pull together what the children have learnt in this session.

\keyquestion{How did you decide which package to \keyword{deliver} first?}

\keyquestion{Can we think of an algorithm for the driver so that the van always takes the shortest route?}

At this level we are just looking for a simple logical approach e.g.\ start at the closest house and then go to the next closest, etc.

\end{lessonplan}

\end{document}
