\documentclass{../../../lessonplan}
\renewcommand{\cflroot}{../../..}

\begin{document}

\lessonplantitle
    {KS1-S10}
    {Key Stage 1 Session 10}
    {Try out a partner's route}

\preamble
    {
    \item Complete a programming challenge set by a peer
    \item Use sequence and repetition independently
    \item Evaluate and debug their program independently
    }
    {
    \item KS1 Self-Assessment sheet KS1-SA
    \vspace{1cm}
    }
    {
    \item Evaluate, progress
    \item Code skills, logical thinking
    \item Computer scientist, programmer
    }

\begin{lessonplan}

\fig{fig S10.1}{figS10.1.jpg}{1}

Choose a child to launch Rapid Router, load their saved challenge and then explain what their partner has to do.

\section*{Individual activity}

The children try out their partner's challenge.

They should complete their Self-Assessment sheets \textit{[fig S.10.1]} to assess their learning.
More advanced children can do this electronically and take a screenshot of the completed route and code, adding this to their Self-Assessment sheet.

\section*{Share and review}

Share a few challenges, with children taking turns to present their work.

\keyquestion{Can you explain how your program works?}

\keyquestion{Can you explain what your code means?}

\keyquestion{What types of programming have you done?}

\keyquestion{How is the computer making the van or character move?}

\keyquestion{What have you learnt, using this app?}

\textbf{Option:} Use a screen recorder, such as the Interactive Whiteboard Smart Recorder, to record the van moving along a child's programmed route.
Use this video in a class presentation or e-book about the project.

\end{lessonplan}

\end{document}
